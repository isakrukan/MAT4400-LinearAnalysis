\section{Integration of Complex Functions (Lec. 7)}
Assume \((X, \mathfrak{B}, \mu)\) is a measure space.
\begin{definition}
    A measurable function \(f:X\rightarrow \mathbb{C}\) is called integrable (or $\mu$-integrable) if
    \begin{align*}
        \int\limits_{X}|f|d\mu < \infty.
    \end{align*}
\end{definition}
Denote by \(\mathcal{L}^1(X, \mathfrak{B}, d\mu)\), \(\mathcal{L}^1(X, d\mu)\) or \(\mathcal{L}_{\mathbb{C}}^{1}\) the set of integrable
functions. This is also a vector space over $\mathbb{C}$, since
\begin{align*}
    |f + g| \leq |f| + |g|, \ |cf| = |c||f| \ (c\in \mathbb{C}), \\
    \text{and the other axioms are should be easy}.
\end{align*}
This vector space is spanned by positive functions, since
\begin{align*}
    f = \text{Re}(f)_{+} - \text{Re}(f)_{-} + i\text{Im}(f)_{+} - i\text{Im}(f)_{-},
\end{align*}
where for a function $h$ we let
\begin{align*}
    h_{+} = \max\left\{h,0\right\}, \ h_{-} = -\min\left\{ h,0 \right\},
\end{align*}
and if \(f\in \mathcal{L}^{1}(X, d\mu)\), then
\begin{align*}
    (\text{Re}(f))_{\pm}, (\text{Im}(f))_{\pm} \in \mathcal{L}^{1}(X, d\mu),
\end{align*}
as
\begin{align*}
    |(\text{Re}(f))_{\pm}|, |(\text{Im}(f))_{\pm}| \leq |f|.
\end{align*}

\begin{prop}
    The integral extends uniquely from the positive integrable functions to a linear function (functional?) \(\mathcal{L}^{1}(X, d\mu) \rightarrow \mathbb{C}\),
    that is, to a map s.t.
    \begin{align*}
        \int\limits_{X}(f + g)d\mu = \int\limits_{X}fd\mu + \int\limits_{X}gd\mu, \\
        \int\limits_{X}cfd\mu = c \int\limits_{X}fd\mu, \ c\in\mathbb{C}.
    \end{align*}
\end{prop}
\begin{proof}
    Uniqueness is clear, as positive functions in \(\mathcal{L}^{1}(X, d\mu)\) spans the entire space. We first extend the integral to real
    integrable functions by letting
    \begin{align*}
        \int\limits_{X}(g-h)d\mu \eqdef \int\limits_{X}gd\mu - \int\limits_{X}hd\mu, 
    \end{align*}
    for \(g,h\in \mathcal{L}^{1}(X,d\mu), \ g,h\geq0\).

    This is well-defined, since if
    \begin{align*}
        g - h = g' - h',
    \end{align*}
    then \(g+h' = h + g'\) and hence \(\int\limits_{X}gd\mu + \int\limits_{X}h'd\mu = \int\limits_{X}g'd\mu + \int\limits_{X}h'd\mu\).
    Now we extend the integral to the entire space \(\mathcal{L}'(X,d\mu)\) by
    \begin{align*}
        \int\limits_{X}fd\mu \eqdef \int\limits_{X} (\text{Re}(f))d\mu + i \int\limits_{X}(\text{Im}(f))d\mu.
    \end{align*}
    We easily get that by definition:
    \begin{align*}
        \int\limits_{X}(f_1+f_2)d\mu = \int\limits_{X}f_1d\mu + \int\limits_{X}f_2d\mu, \ \forall f_1, f_2 \in \mathcal{L}^{1}(X, d\mu),
    \end{align*}
    and
    \begin{align*}
        \int\limits_{X}cfd\mu = c\int\limits_{X}fd\mu \ \forall f\in \mathcal{L}^{1}(X,d\mu) \ \forall c\geq 0.
    \end{align*}

    In order to prove the last property for all \(c\in\mathbb{C}\), it remains to check it for \(c=-1\) and \(c=i\).

    For \(c=-1\) it follows, since if \(g,h\geq0\), then
    \begin{eqnarray*}
        \int\limits_{X}\left( -(g-h) \right)d\mu &=& \int\limits_{X}\left( h-g \right)d\mu \\
        &=& \int\limits_{X} h d\mu - \int\limits_{X} gd\mu \\
        &=& - \int\limits_{X}\left( g- h \right)d\mu.
    \end{eqnarray*}
    Similarly, for \(c=i\) it is proved by a simple computation:
    \begin{eqnarray*}
        \int\limits_{X} i fd\mu &=& \int\limits_{X} \text{Re}(if)d\mu + i\int\limits_{X}\text{Im}(if)d\mu \\
        &=& \int\limits_{X}\left( -\text{Im}(f) \right)d\mu + i \int\limits_{X} \left( \text{Re}(f) \right)d\mu \\
        &=& i\left( \int\limits_{X}\left( \text{Re}(f) \right) d\mu + i\int\limits_{X} \left( \text{Im}(f) \right)d\mu\right) \\
        &=& i\int\limits_{X} fd\mu.
    \end{eqnarray*}
\end{proof}
\begin{prop}[Triangle Inequality]
    For every \(f\in \mathcal{L}^{1}(X,d\mu)\) we have
    \begin{align*}
        \Big|\int\limits_{X}fd\mu\Big| \leq \int\limits_{X}|f|d\mu.
    \end{align*}
\end{prop}
\begin{proof}
    Choose \(z\in \Pi\eqdef \left\{ w\in\mathbb{C} : |w|=1 \right\}\) s.t.
    \begin{align*}
        z\int\limits_{X}fd\mu \geq 0.
    \end{align*}
    Then
    \begin{eqnarray*}
        \Big|\int\limits_{X} fd\mu \Big| &=& \Big|z\int\limits_{X}\Big| \\
        &=& z \int\limits_{X}fd\mu \\
        &=& \int\limits_{X}zfd\mu \\
        &=& \int\limits_{X}\text{Re}(zf)d\mu + i \cancel{\int\limits_{X}\text{Im}(zf)d\mu} \\
        &=& \int\limits_{X}\left(\text{Re}(zf)\right)_{+}d\mu - \int\limits_{X}\left(\text{Re}(zf)\right)_{-}d\mu \\
        &\leq& \int\limits_{X}\left(\text{Re}(zf)\right)_{+}d\mu \\
        &\leq& \int\limits_{X}|f|d\mu,
    \end{eqnarray*}
    since \(\left( \text{Re}(zf) \right)_{+} \leq |f|\).
\end{proof}

\section{Completions of Measure Spaces (Lec. 8)}
\begin{definition}
    A measure space \((X,\mathfrak{B}, \mu)\) is called \textbf{complete} if whenever \(A\in\mathfrak{B}, \mu(A)=0\), we have
    \(B\in\mathfrak{B}\) for all \(B\subset A\).
\end{definition}

Any measure space \((X, \mathfrak{B},\mu)\) can be completed as follows:

Let \(\bar{\mathfrak{B}}\) be the \(\sigma\)-algebra generated by \(\mathfrak{B}\) and all sets \(B\subset X\) s.t. \(\exists A\in\mathfrak{B}\)
with \(B\subset A\) and \(\mu(A) = 0\).

\begin{prop}
    The \(\sigma\)-algebra \(\bar{\mathfrak{B}}\) can also be described as follows:
    \begin{align*}
        \bar{\mathfrak{B}} \eqdef \left\{ B\subset X : A_1\subset B \subset A_2 \text{ for some } A_1,A_2\in \mathfrak{B} \text{ with }
        \mu(A_2\backslash A_1)\right\},
    \end{align*}
    with \(B, A_1, A_2\) as above, we define 
    \begin{align*}
        \bar{\mu}(B) \eqdef \mu(A_1) = \mu(A_2).
    \end{align*}
    Then \((X,\bar{\mathfrak{B}},\bar{\mu})\) is a complete measure space.
\end{prop}
\begin{proof}
    Consider the collection of sets
    \begin{align*}
        \tilde{\mathfrak{B}} \eqdef \left\{ B\subset X : A_1\subset B \subset A_2 \text{ for some } A_1,A_2\in \mathfrak{B} \text{ with }
        \mu(A_2\backslash A_1)\right\},
    \end{align*}
    this is a \(\sigma\)-algebra; \(\emptyset \in \tilde{\mathfrak{B}}\) (take \(A_1=A_2=\emptyset\)), if \(B\in\tilde{\mathfrak{B}}\) and
    \(A_1,A_2\) are as above, then \(A_{2}^{c}\subset B^{c}\subset A_{1}^{c}\) and \(\mu(A_1\backslash A_2) = \mu(A_2\backslash A_1) = 0\), so 
    \(B^{c}\in\tilde{\mathfrak{B}}\). Finally, assume \(\left(  B_n\right)_{n\in\mathbb{N}} \subset \tilde{\mathfrak{B}}\) and let
    \(A'_{n}, A''_{n}\in\mathfrak{B}\) be s.t.
    \begin{align*}
        A'_n\subset B_n\subset A''_n, \ \mu(A''_n\backslash A'_n) = 0.
    \end{align*}
    Put \(A_1 = \bigcup\limits_{n=1}^{\infty}A'_n, \ A_2 = \bigcup\limits_{n=1}^{\infty}A''_n\). Then
    \begin{align*}
        A_1\subset \bigcup\limits_{n=1}^{\infty}B_n\subset A_2, \ A_2\backslash A_1 \subset \bigcup\limits_{n=1}^{\infty}\left( A''_n\backslash A'_n \right),
    \end{align*}
    so
    \begin{align*}
        \mu(A_2\backslash A_1) \leq \sum\limits_{n=1}^{\infty}\mu(A''_n\backslash A'_n),
    \end{align*}
    hence
    \begin{align*}
        \bigcup\limits_{n=1}^{\infty} B_n \in \tilde{\mathfrak{B}}.
    \end{align*}
    Thus, \(\tilde{\mathfrak{B}}\) is a \(\sigma\)-algebra. It contains \(\mathfrak{B}\) and all sets \(B\subset X\) s.t. \(B\subset A\)
    and \(\mu(A)=0\) for some \(A\in \mathfrak{B}\). Hence, \(\bar{\mathfrak{B}}\subset \tilde{\mathfrak{B}}\). On the other hand, if 
    \(B\in \tilde{\mathfrak{text}}\), \(A_1 \subset B\subset A_2\), \(A_1, A_2 \in \mathfrak{B}\) and \(\mu(A_2\backslash A_1) = 0\), then
    \begin{align*}
        B=A_1\cup \left(  B\backslash A_1\right) \text{ and } B\backslash A_1 \subset A_2\backslash A_1,
    \end{align*}
    hence \(B\in \bar{\mathfrak{B}}\). Thus \(\bar{\mathfrak{B}} = \tilde{\mathfrak{B}}\).

    Next, we need to show that \(\bar{\mu}\) is well-defined. Assume
    \begin{align*}
        A_1\subset B\subset A_2, \ A'_1\subset B \subset A'_2,
    \end{align*}
    with \(A_1, A_2, A'_1, A'_2\in\mathfrak{B}, \ \mu(A_2\backslash A_1) = \mu(A'_2\backslash A'_1) = 0\). Then
    \begin{align*}
        A_1\backslash A'_1\subset A'_2\backslash A'_1,
    \end{align*}
    hence
    \begin{align*}
        \mu(A_1\backslash A'_1) = 0.
    \end{align*}
    It follows that
    \begin{align*}
        \mu(A_1) = \mu(A_1\cap A'_1),
    \end{align*}
    and for the same reason that
    \begin{align*}
        \mu(A'_1) = \mu(A_1\cap A'_1).
    \end{align*}
    Therefor \(\mu(A_1) = \mu(A'_1)\), so \(\bar{\mu}\) is well-defined.

    Finally, we have to check that \(\mu\) is a measure. Assume \(\left( B_n \right)_{n\in\mathbb{N}}\) is a sequence of disjoint sets in 
    \(\bar{\mathfrak{B}}\). As above, choose \(A'_n, A''_n\in \mathfrak{B}\) s.t.
    \begin{align*}
        A'_n \subset B_n \subset A''_n \text{ and } \mu(A''_n\backslash A'_n) = 0.
    \end{align*}
    Then for \(A_1 = \bigcup\limits_{n=1}^{\infty} A'_n\) and \(A_2 = \bigcup\limits_{n=1}^{\infty} A''_n\) we have
    \begin{align*}
        A_1 \subset \bigcup\limits_{n=1}^{\infty} B_n\subset A_2 \text{ and } \mu(A_2\backslash A_1) = 0.
    \end{align*}
    Hence,
    \begin{eqnarray*}
        \bar{\mu} \left( \bigcup\limits_{n=1}^{\infty}B_n \right) &=& \mu(A_1) = \mu\left( \bigcup\limits_{n=1}^{\infty} A'_n \right) \\
        = \sum\limits_{n=1}^{\infty} \mu(A'_n) &=& \sum\limits_{n=1}^{\infty}\bar{\mu}(B_n).
    \end{eqnarray*}
    We also have \(\bar{\mu}(\emptyset) = \mu(\emptyset) = 0\).
\end{proof}
\begin{definition}
    If \(\mu\) is a Borel measure on a \textbf{metric} space \((X,d)\), then the completion \(\mathfrak{B}(X)\) of the Borel \(\sigma\)-algebra
    with respect to \(\mu\) is called the \(\sigma\)-algebra of \(\mu\)-measurable sets.
\end{definition}

For \(\mu = \lambda_n\) on \(\mathbb{R}^n\) we talk about the \(\sigma\)-algebra of \textbf{Lebesque measurable sets}. Instead of 
\(\bar{\lambda}_n\) we still write \(\lambda_n\) and call it the \textbf{Lebesgue measure}. A function \(f: \mathbb{R}^n\rightarrow \mathbb{C}\),
measurable with respect to the \(\sigma\)-algebra of Lebesgue measurable sets is called \textbf{Lebesgue measurable}.

\begin{prop}
    Assume \((X, \mathfrak{B}, \mu)\) is a measure space and consider its completion \((X, \bar{\mathfrak{B}}, \bar{\mu})\). Assume 
    \(f: X\rightarrow \mathbb{C}\) is \(\bar{\mathfrak{B}}\)-measurable. Then there is a \(\mathfrak{B}\)-measurable function
    \(g:X\rightarrow \mathbb{C}\) s.t. \(f=g \ \bar{\mu}\)-a.e (almost everywhere).
\end{prop}
\begin{proof}
    By considering separately \(\left(\text{Re}(f)\right)_{\pm}\), \(\left(\text{Im}(f)\right)_{\pm}\) we may assume that \(f\geq 0\).
    Assume first that \(f\) is simple, 
    \begin{align*}
        f = \sum\limits_{k=1}^{\infty} c_k1_{B_k}, \ c_k \geq 0, \ B_k\in\bar{\mathfrak{B}}.
    \end{align*}
    Let \(A_k\in \mathfrak{B}\) be s.t. \(A_K\subset B_k\), \(\bar{\mu}(B_k\backslash A_k) = 0\). Then define
    \begin{align*}
        g = \sum\limits_{k=1}^{\infty}c_k 1_{A_k}.
    \end{align*}
    We have \(0\leq g\leq f\) and \(f=g \ \bar{\mu}\)-a.e., namely,
    \begin{align*}
        \left\{ x:f(x)\neq g(x) \right\} \subset \bigcup\limits_{n=1}^{n}\left(B_k\backslash A_k\right).
    \end{align*}
    For general \(f\geq 0\) choose simple \(\bar{\mathfrak{B}}\)-measurable functions \(f_n\) s.t.
    \begin{align*}
        0\leq f_1\leq f_2\leq ..., \ f_n \uparrow f \text{ pointwise.}
    \end{align*}
    Then we can find simple \(\mathfrak{B}\)-measurable functions \(g_n\) s.t. \(0\leq g_n\leq f_n\) and the set 
    \begin{align*}
        A_n \eqdef \left\{ x:f_n(x)\neq g_n(x) \right\}
    \end{align*}
    has measure zero. Consider
    \begin{align*}
        \tilde{g}_n \eqdef \max\left\{ g_1, ..., g_n \right\}.
    \end{align*}
    Then
    \begin{align*}
        0\leq \tilde{g}_1\leq \tilde{g}_2 \leq ..., \ \tilde{g}_n \leq f_n,
    \end{align*}
    and 
    \begin{align*}
        \left\{ x: \tilde{g}_n(x) < f_n(x) \right\} \subset A_n.
    \end{align*}
    Define \(g(x) \eqdef \lim\limits_{n\rightarrow \infty} \tilde{g_n}(x)\). Then \(g(x)\) is \(\mathfrak{B}\)-measurable, \(g\leq f\) and
    \begin{align*}
        \left\{ x : g(x) < f(x) \right\} \subset \bigcup\limits_{n=1}^{\infty} A_n,
    \end{align*}
    so \(g=f \ \bar{\mu}\)-a.e.
\end{proof}
