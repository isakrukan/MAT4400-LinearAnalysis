\documentclass{article}


% allows special characters (including æøå)
\usepackage[utf8]{inputenc}
%\usepackage[english]{babel}

\usepackage{subfiles}
\usepackage{dsfont}
\usepackage{physics,amssymb}  % mathematical symbols (physics imports amsmath)
\include{amsmath}
\usepackage{graphicx}         % include graphics such as plots
\usepackage{xcolor}           % set colors
\usepackage{hyperref}         % automagic cross-referencing (this is GODLIKE)
\usepackage{listings}         % display code
\usepackage{subfigure}        % imports a lot of cool and useful figure commands
\usepackage{float}
%\usepackage[section]{placeins}
\usepackage{algorithm}
\usepackage[noend]{algpseudocode}
\usepackage{subfigure}
\usepackage{tikz}
\usepackage{cleveref} % for \cref
\usepackage{enumitem} % to enumerate with a), b), ... : [label=(\alph*)] 
\usepackage{cancel}
\usepackage{slashed}
\usepackage{amsthm}
\usepackage{mathrsfs}





\newtheorem{theorem}{Theorem}[section]
\newtheorem{properties}[theorem]{Properties}
\newtheorem{lemma}[theorem]{Lemma}
\newtheorem{corollary}[theorem]{Corollary}
\newtheorem{proposition}[theorem]{Proposition}
\newtheorem*{remark}{Remark}
\newcommand{\eqdef}{\mathrel{\mathop:}=}
\theoremstyle{definition}
\newtheorem{definition}[theorem]{Definition}
\usepackage{stmaryrd}

\usetikzlibrary{quantikz}

% defines the color of hyperref objects
% Blending two colors:  blue!80!black  =  80% blue and 20% black


\newcommand{\A}{\mathscr{A}}
\newcommand{\R}{\mathbb{R}}
\newcommand{\C}{\mathbb{C}}
\newcommand{\B}{\mathscr{B}}
\newcommand{\Borel}[1]{\mathscr B(\mathbb R^{#1})}
\newcommand{\omm}{\hspace{1mm}}
\newcommand{\tmm}{\hspace{2mm}}
\newcommand{\map}[3]{#1:#2\rightarrow #3}



\lstset{frame=tb,
  language=c++,
  aboveskip=3mm,
  belowskip=3mm,
  showstringspaces=false,
  columns=flexible,
  basicstyle={\small\ttfamily},
  numbers=none,
  numberstyle=\tiny\color{gray},
  keywordstyle=\color{blue},
  commentstyle=\color{dkgreen},
  stringstyle=\color{orange},
  breaklines=true,
  breakatwhitespace=true,
  tabsize=4
}
\hypersetup{
    colorlinks,
    linkcolor={red!50!black},
    citecolor={blue!50!black},
    urlcolor={blue!80!black}}
%% USEFUL LINKS:
%%
%%   UiO LaTeX guides:        https://www.mn.uio.no/ifi/tjenester/it/hjelp/latex/
%%   mathematics:             https://en.wikibooks.org/wiki/LaTeX/Mathematics

%%   PHYSICS !                https://mirror.hmc.edu/ctan/macros/latex/contrib/physics/physics.pdf

%%   the basics of Tikz:       https://en.wikibooks.org/wiki/LaTeX/PGF/Tikz
%%   all the colors!:          https://en.wikibooks.org/wiki/LaTeX/Colors
%%   how to draw tables:       https://en.wikibooks.org/wiki/LaTeX/Tables
%%   code listing styles:      https://en.wikibooks.org/wiki/LaTeX/Source_Code_Listings
%%   \includegraphics          https://en.wikibooks.org/wiki/LaTeX/Importing_Graphics
%%   learn more about figures  https://en.wikibooks.org/wiki/LaTeX/Floats,_Figures_and_Captions
%%   automagic bibliography:   https://en.wikibooks.org/wiki/LaTeX/Bibliography_Management  (this one is kinda difficult the first time)
%%   REVTeX Guide:             http://www.physics.csbsju.edu/370/papers/Journal_Style_Manuals/auguide4-1.pdf
%%
%%   (this document is of class "revtex4-1", the REVTeX Guide explains how the class works)


%%%%%%%%%%%%% CREATING THE pdf FILE USING LINUX IN THE TERMINAL %%%%%%%%%%%%%
%% pdflatex filename.tex && filename.tex && open filename.pdf

%%%%%%%%%%%%%%%%%%%%%%%%%% USING FOOTNOTE COMMAND: %%%%%%%%%%%%%%%%%%%%%%%%%%
    %% pdflatex filename.tex && bibtex filename.tex && pdflatex filename.tex && pdflatex filename.tex && open filename.pdf


\setcounter{section}{7}
\setcounter{theorem}{7}


\begin{document}
\title{Regularity of measures}
\maketitle
\date

\section{Regularity of measures}

We let $(X,d)$ be a metric space and denote by $\mathcal O$ the open, by $\mathcal C$ the closed and $\B(X)=\sigma(\mathcal O)$ the Borel set of $X$.

\begin{definition}
    A measure $\mu$ on $(X,d,\B(X))$ is called outer regular, if \begin{align}
        \mu(B) = \inf\left\{\mu(U)\omm|\omm B\subset U,\omm U \text{ open} \right\}
    \end{align}
    and inner regular, if $\mu(K)<\infty$ for all compact sets $K\subset X$ and \begin{align}
        \mu(U) = \sup\left\{\mu(K)\omm|\omm K\subset U,\omm K \text{ compact} \right\}.
    \end{align}
    A measure which is both inner and outer regular is called \textbf{regular}. We write $\mathfrak m_r^+(X)$ for the family of 
    regular measures on $(X,\B(X))$.
\end{definition}

\begin{remark}
    The space $X$ is called $\sigma$-compact if there is a sequence of compact sets $K_n\uparrow X$. A typical example of such a space 
    is a locally compact, separable metric space.
\end{remark}

\begin{theorem}
    Let $(X,d)$ be a metric space. Every finite measure $\mu$ on $(X,\B(X))$ is outer regular. If $X$ is $\sigma$-compact, then $\mu$ is also inner regular, hence regular.
\end{theorem}
\begin{theorem}
    Let $(X,d)$ be a metric space and $\mu$ be a measure on $(X,B(X))$ such that $\mu(K)<\infty$ for all compact sets $K\subset X$.
\end{theorem}
\begin{itemize}
    \item[1] If $X$ is $\sigma$-compact, then $\mu$ is inner regular.
    \item[2] If there exists a sequence $G_n\in\mathcal O,\omm G_n\uparrow X$ such that $\mu(G_n)<\infty$, then $\mu$ is outer regular.   
\end{itemize}
\end{document}