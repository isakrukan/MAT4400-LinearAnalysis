\section{Integrals of Measurable Functions}
We have defined our integral for positive measurable functions, i.e. functions in \(\mathcal{M}^{+}(\mathscr{A})\). To extend our integral
to not only functions in \(\mathcal{M}^{+}(\mathscr{A})\) we first notice that
\begin{align}
    u \in \highlight[yellow]{\mathcal{M}_{\overline{\mathbb{R}}}(\mathscr{A})} \Leftrightarrow 
    u = u^+ - u^-, \ u^+, u^-\in \highlight[yellow]{\mathcal{M}^{+}_{\overline{\mathbb{R}}}},
\end{align}
i.e. that every measurable function can be written as a sum of \textbf{positive} measurable functions.

\begin{definition}[\(\mu\)-integrable]
    A function \(u:X \rightarrow \overline{\mathbb{R}}\) on \((X, \mathscr{A}, \mu)\) is \(\mu\)-\emph{integrable}, if it is 
    \(\mathscr{A}/\mathscr{B}(\overline{\mathbb{R}})\)-measurable and if \(\int u^+ d\mu, \int u^-d\mu < \infty\) (recall the definition
    for the integral of positive measurable functions). Then
    \begin{align}
        \int ud\mu \eqdef \int u^+d\mu - \int u^-d\mu \in (-\infty, \infty)
    \end{align}
    is the (\(\mu\)-)\emph{integral} of u. We write \(\mathcal{L}^1(\mu)\) for the set of all real-valued \(\mu\)-integrable functions
    \footnote{In words, we extend our integral to \cancel{positive} measurable functions by noticing that we can write every measurable 
    function as a sum of positive measurable functions, something that we do know how to integrate. We don't want to run into the problem
    of \(\infty - \infty\), thus we require the integral of the positive and negative parts to both (separately) be less than infinity.}.
\end{definition}
\begin{theorem}
    Let \(u\in \mathcal{M}_{\overline{\mathbb{R}}}(\mathscr{A})\), then the following conditions are equivalent:
    \begin{enumerate}[label=(\roman*)]
        \item \(u \in \mathcal{L}^{1}_{\overline{\mathbb{R}}}(\mu)\).
        \item \(u^+, u^- \in \mathcal{L}^{1}_{\overline{\mathbb{R}}}(\mu)\).
        \item \(\vert u\vert \in \mathcal{L}^{1}_{\overline{\mathbb{R}}}(\mu)\).
        \item \(\exists w \in \mathcal{L}^{1}_{\overline{\mathbb{R}}}(\mu)\) with \(w\geq 0\) s.t. \(\vert u \vert \leq w\).
    \end{enumerate}
\end{theorem}
\begin{theorem}[Properties of the \(\mu\)-integral] The \(\mu\)-integral is: \textbf{homogeneous, additive}, and:
    \begin{enumerate}[label=(\roman*)]
        \item \(\min\left\{u,v\right\}, \max\left\{u,v\right\} \in \mathcal{L}^{1}_{\overline{\mathbb{R}}}(\mu)\) \itemtag{\emph{lattice property}}
        \item \(u\leq v \Rightarrow \int ud\mu \leq \int vd\mu\) \itemtag{\emph{monotone}}
        \item \(\Big\vert \int ud\mu \Big\vert \leq \int \vert u\vert d\mu\) \itemtag{\emph{triangle inequality}}
    \end{enumerate}
\end{theorem}
\begin{remark}
    If \(u(x) \pm v(x)\) is defined in \(\overline{\mathbb{R}}\) for all \(x\in X\) then we can exclude \(\infty - \infty\) and the theorem above just says that 
    the integral is linear:
    \begin{align}
        \int (au + bv)d\mu = a\int ud\mu + b\int vd\mu.
    \end{align}
    This is always true for real-valued \(u,v\in\mathcal{L}^{1}(\mu) = \mathcal{L}^{1}_{\mathbb{R}}(\mu)\), making \(\mathcal{L}^{1}(\mu)\) a vector space with 
    addition and scalar multiplication defined by
    \begin{align}
        (u + v)(x) \eqdef u(x) + v(x), \ (a\cdot u)(x) \eqdef a\cdot u(x),
    \end{align}
    and
    \begin{align}
        \int ... d\mu: \mathcal{L}^{1}(\mu) \rightarrow \mathbb{R}, \ u \mapsto \int ud\mu,
    \end{align}
    is a \textbf{positive linear functional}.
\end{remark}

