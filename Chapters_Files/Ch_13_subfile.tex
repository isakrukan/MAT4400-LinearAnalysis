\section{The Function Spaces $\mathcal L^p$}
Assume $V$ is a vector space over $\mathbb K\in\{\mathbb C,\mathbb R\}$.

\begin{definition}
    A seminorn on $V$ is a map $\map{p}{V}{[0,+\infty)}$ s.t. 
    \begin{itemize}
        \item[(1)] $p(cx)=|c|p(x)\tmm \forall x\in V,\forall c\in \mathbb K$.
        \item[(2)] $p(x+y)\leq p(x) +p(y)\tmm \forall x,y\in V. \hspace{4mm} \textbf{triangle inequality}. $
    \end{itemize}
    A seminorm is called a norm if we also have \begin{align*}
        p(x)=0\tmm \iff \tmm x=0.
    \end{align*}
\end{definition}

A norm is commonly denoted $||x||$, and a vectorspace equipped with a norm is called a \textbf{normed space}.


\begin{definition}
    Assume $(X,d)$ is a measure space. Fix $1\leq p\leq \infty.$ For every measurable function $\map{f}{X}{\C}$ we define the following
    \begin{align}
        ||f||_p = \left( \int_X |f|^p d\mu\right)^{1/p}\in [0,+\infty].
    \end{align} 
    We can see that $||cf||_p=|c|||f||_p\tmm \forall c\in\C.$
\end{definition}

\begin{lemma}
    \begin{align}
        ||f+g||_p\leq ||f||_p+||g||_p.
    \end{align}
\end{lemma}

\begin{definition}
    We define
    \begin{align}
        \mathcal L^p(X,d\mu) = \left\{ \map{f}{X}{\C}\tmm |\tmm f\text{ is measurable and } ||f||_p<\infty \right\}.    
    \end{align}
    This is a vectorspace with seminorm $f\mapsto ||f||_p.$
     And in general this is not a normed space, since $||f||_p=0 \iff f=0 \text{ a.e.}$
\end{definition}

Generally, if $p$ is a seminorm on a vectorspace $V$, then
 \begin{align}
    V_0 = \left\{x\in V\tmm|\tmm p(x)=0 \right\}
\end{align}
which is a subspace of $V$. Then we consider the quotient/factor space $V/V_0.$

\begin{definition}
    For $x,y\in V$, define \begin{align}
        x\sim y \iff x-y\in V_0.
    \end{align}
    This is an equivalence relation on $V$. The representation class of $V$ is defined by $[x]$ or $x+V_0$.
\end{definition}

Then $V/V_0$ is equals the set of equivalence classes. We can show that it is a normed space. 

$$[x]+[y]=[x+y]\tmm,\tmm c[x]=[cx]\tmm,\tmm ||[x]||=p(x). $$

Applying this to $\mathcal L^p(X,d\mu)$ we get the normed space \begin{align}
    L^p(X,d\mu) = \mathcal L ^p(X,d\mu)/\mathcal N.
\end{align}
Where $\mathcal N$ is the space of measurable functions $f$ s.t. $f=0$ a.e. We will further continue to denote the norm by $||\cdot||_p$, and we will normally \textbf{not} distinguish between $f\in \mathcal L^p(X,d\mu)$ and the vector in 
$L^p(X,d\mu)$ that $f$ defines.

\begin{definition}
    A normed space $(X,||\cdot||)$ is called a Banach space if $V$ is complete w.r.t the metric $d(x,y)=||x-y||$.
\end{definition}

\begin{theorem}
    If $(X,\B,\mu)$ is a measure space, $1\leq p\leq \infty$, then $L^p(X,d\mu)$ is a Banach space.
\end{theorem}

\begin{definition}
    A measurable function $\map{f}{X}{\C}$ is called \textbf{essentially bounded} if there is $c\geq 0$ s.t. 
    \begin{align}
    \mu(\left\{ x\omm:\omm |f(x)|>c \right\})=0.    
    \end{align}
    That is $|f|\leq c$ a.e. The smallest such $c$ is called the essential supremum of $f$ and is denoted by $||f||_\infty.$
\end{definition}

\begin{definition}
    $$\mathcal L^\infty(X,d\mu) = \left\{ \map{f}{X}{\C}\tmm|\tmm f\text{ is measurable and }||f||_\infty <\infty \right\}.$$
    $$L^\infty (X,d\mu) =\mathcal L^\infty(X,d\mu)/\mathcal N. $$
    Where by the previous definiton these spaces become the spaces of all essentially bounded functions. 
\end{definition}
\begin{theorem}
    If $(X,\B,\mu)$ is a $\sigma$-finite measure space, then $L^\infty(X,d\mu)$ is a Banach space.
\end{theorem}