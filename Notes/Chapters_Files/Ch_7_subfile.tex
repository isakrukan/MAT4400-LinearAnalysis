\section{Measurable Mappings}

    We consider maps $T:X\rightarrow X'$ between two measurable spaces $(X,\mathcal A)$ and $(X',\mathcal A')$ which respects the measurable 
    structurs, the $\sigma$-algbras on $X$ and $X'$. These maps are useful as we can transport a measure $\mu$, defined on $(X,\mathcal A)$, to $(X',\mathcal A')$.

    

    \begin{definition}
        Let $(X,\mathcal A)$, $(X',\mathcal A')$ b measurable spaces. A map $T:X\rightarrow X'$ is called $\mathcal A/\mathcal A'$-measurable if the pre-imag of every measurable set is a measurable set: 
        \begin{align}
        T^{-1}(A')\in \mathcal A,\hspace{3mm} \forall A'\in\mathcal A'.    
        \end{align}

    \end{definition}
    \begin{itemize}
        \item A $\mathcal B(\mathbb R^n)/\mathcal B(\mathbb R^m)$ measurable map is often called a Borel map. 
        \item The notation $T:(X,\mathcal A)\rightarrow (X',\mathcal A')$ is often used to indicate measurability of the map $T$.
    \end{itemize}

    \begin{lemma}
        Let $(X,\mathcal A)$, $(x',\mathcal A')$ be measurable spaces and let $\mathcal A' = \sigma(\mathcal G')$. Then $T:X\rightarrow X'$ is $\mathcal A/\mathcal A'$-measurable iff $T^{-1}(\mathcal G')\subset \mathcal A$, i.e. if 
        \begin{align}
            T^{-1}(G')\in\mathcal A,\hspace{2mm} \forall G'\in\mathcal G'.
        \end{align} 
    \end{lemma}


	\begin{theorem}

        Let $(X_i,\mathcal A_i),\hspace{2mm}i=1,2,3$, be measurable spaces and $T:X_1\rightarrow X_2$, $S:X_2\rightarrow X_3$ be $\mathcal A_1/\mathcal A_2$ and $\mathcal A_2/\mathcal A_3$-measurable maps respectivly. Then $S\circ T:X_1\rightarrow X_3$ is $\mathcal A_1/\mathcal A_3$-measurable. 

	\end{theorem}

    \begin{corollary}
        Every continuous map betwen metric spaces is a Borel map.
    \end{corollary}

    \begin{definition}
        \textbf{(and lemma)}
        Let $(T_i)_{i\in I}$, $T_I:X\rightarrow X_i$, be arbitrarily many mappings from the same space X into measurable spaces $(X_i,\mathcal A_i).$ The smallest $\sigma$-algebra on $X$ that makes all $T_i$ simultanously measurable is 
        \begin{align}
            \sigma(T_i:i\in I) := \sigma \left( \bigcup_{i\in I}T_i^{-1}(\mathcal A_i)\right)
        \end{align}

    \end{definition}
	
    \begin{corollary}
        A function $f:(X,\mathcal B)\rightarrow \mathbb R$ is measurable if $f((a,+\infty))\in \mathcal B,\hspace{2mm}\forall a\in \mathbb R.$
    \end{corollary}

    \begin{corollary}
        Assume $(X,\mathcal B)$ is a measurable space, $(Y,d)$ is a metric space, $(f_n:(X,\mathcal B)\rightarrow Y)_{n=1}^\infty$ is a sequence of measurable maps.
        Assume this sequence of images $(f_n(x))_{n=1}^\infty$ is convergent in $Y$ $\forall x\in X$.
        Define \begin{align}
            f:X\rightarrow Y,\hspace{2mm}\text{by  } f(x)=\lim_{n\rightarrow \infty} f_n(x).
        \end{align}
        Then f is measurable.
    \end{corollary}

    \begin{theorem}
        Let $(X,\mathcal A)$, $(X',\mathcal A')$ be measurable spaces and $T:X\rightarrow X'$ be an $\mathcal A/\mathcal A'$-measurable map. For every measurable $\mu$ on $(X,\mathcal A)$, 
        \begin{align}
            \mu'(A') := \mu (T^{-1}(A')),\hspace{3mm} A'\in \mathcal A',
        \end{align}
        defines a measure on $(X',\mathcal A')$.
    \end{theorem}

    \begin{definition}
        The measure $\mu'(\cdot)$ in the above theorem is called the push forward or image measure of $\mu$ under $T$ and it is denoted as 
        $T(\mu)(\cdot)$, $T_{*\mu}(\cdot)$ or $\mu\circ T^{-1}(\cdot)$.
    \end{definition}

    \begin{theorem}
        If $T\in \mathbb R^{n\times n} $ is an orthogonal matrix, then $\lambda^n = T(\lambda^n).$
    \end{theorem}

    \begin{theorem}
        Let $S\in\mathbb R^{n\times n}$ be an invertible matrix. Then 
        \begin{align}
        S(\lambda^n) = |\det s^{-1}|\lambda^n = |\det S|^{-1}\lambda^n.    
        \end{align}
    \end{theorem}

    \begin{corollary}
        Lebesgue measure is invariant under motions: $\lambda^n=M(\lambda^n)$ for all motions $M$ in $\mathbb R^n$. In particular, congruent sets have the same measure. Two sets of points are called congruent if, and only if, one can be transformed into the other by an isometry
    \end{corollary}