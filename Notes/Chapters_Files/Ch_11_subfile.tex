\section{Null sets and the Almost Everywhere (lecture 08, 05. Feb.)}
\begin{definition}
    A (\(\mu\)-)null set \(N\in\mathcal{N}_{\mu}\) is a measurable set \(N\in\mathscr{A}\) satisfying
    \begin{align}
        N\in\mathcal{N}_{\mu} \Longleftrightarrow N\in\mathscr{A} \text{ and } \mu(N) = 0.
    \end{align}
    This can be used generally about a `statement' or `property', but we will be interested in questions like 
    `when is \(u(x)\) equal to \(v(x)\)', and we answer this by saying
    \begin{align}
        u=v \ a.e. \Leftrightarrow \left\{ x: u(x) \neq v(x) \right\} \text{ is (contained in) a }\mu\text{-null set.}, 
    \end{align}
    i.e.
    \begin{align}
        u=v \ \  \mu\text{-a.e.} \Leftrightarrow \mu\left( \left\{ x: u(x) \neq v(x) \right\} \right) = 0.
    \end{align}
    The last phrasing should of course include that the set \( \left\{ x: u(x) \neq v(x) \right\}\) is in \(\mathscr{A}\).
\end{definition}
\begin{theorem} \label{th:11.2}
    Let \(u\in \mathcal{M}_{\overline{\mathbb{R}}}(\mathscr{A})\), then:
    \begin{enumerate}[label=(\roman*)]
        \item \(\int\vert u\vert d\mu = 0 \Leftrightarrow \vert u\vert = 0 \text{ a.e. } \Leftrightarrow \mu\left\{u\neq0\right\} = 0\),
        \item \(\mathbbm{1}_{N}u\in \mathcal{L}^{1}_{\overline{\mathbb{R}}}(\mu)\) \ \(\forall \ N\in\mathcal{N}_{\mu}\),
        \item \(\int_Nud\mu = 0\).
    \end{enumerate}
    (i) is really useful, later we will define \(\mathcal{L}^p\) and the \(||\cdot||_p\)-(semi)norm. Then (i) means that if we have a
    sequence \(u_n\) converging to \(u\) in the \(||\cdot ||_p\)-norm then \(u_n(x) = u(x)\) a.e.
\end{theorem}
\begin{corollary}
    Let \(u=v \ \mu \text{-a.e.}\) Then
    \begin{enumerate}[label=(\roman*)]
        \item \(u,v \geq 0 \Rightarrow \int ud\mu = \int vd\mu\),
        \item \(u\in\mathcal{L}^{1}_{\overline{\mathbb{R}}}(\mu) \Rightarrow v\in\mathcal{L}^{1}_{\overline{\mathbb{R}}}(\mu)\) and \(\int ud\mu = \int vd\mu\).
    \end{enumerate}
\end{corollary}
\begin{corollary}
    If \(u\in\mathcal{M}_{\overline{\mathbb{R}}}(\mathscr{A})\), \(v\in\mathcal{L}^{1}_{\overline{\mathbb{R}}}(\mu)\) and \(v\geq0\) then
    \begin{align}
        \vert u\vert \leq v \text{ a.e. } \Rightarrow u\in\mathcal{L}^{1}_{\overline{\mathbb{R}}}(\mu).
    \end{align}
\end{corollary}
\begin{proposition}[Markow inequality]
    For all \(u\in\mathcal{L}^{1}_{\overline{\mathbb{R}}}(\mu), \ A\in\mathscr{A}\) and \(c>0\)
    \begin{align}
        u\left(\left\{ \vert u\vert \geq c \right\} \cap A\right) \leq \frac{1}{c}\int_{A}\vert u\vert d\mu,
    \end{align}
    if \(A=X\), then (obviosly)
    \begin{align}
        u\left\{ \vert u\vert \geq c \right\} \leq \frac{1}{c}\int \vert u\vert d\mu.
    \end{align}
\end{proposition}    
\ifdetailed
\begin{corollary}
    If \(u\in\mathcal{L}^{1}_{\overline{R}}(\mu)\), then \(\mu\) is a.e. \(\mathbb{R}\)-vaued. In particular, we can find a version
    \(\tilde{u}\in \mathcal{L}^{1}(\mu)\) s.t. \(\tilde{u}=u\) a.e. and \(\int\tilde{u}d\mu = \int ud\mu\)
\end{corollary}
\fi

\subsection*{Completions of measure spaces}
\begin{definition}
    A measure space \(\left(X,\mathscr{B}, \mu\right)\) is called \textbf{complete} if whenever \(A\in\mathscr{B} \text{ and } \mu(A) =0\), we have \(B\in\mathscr{B} \ \forall B\subset A\).
\end{definition}
\begin{remark}
    Any measure space can be completed as follows: \\ 
    Let \(\bar{\mathscr{B}}\) be the \(\sigma\)-algebra generated by \(\mathscr{B}\) and all sets \(B\subset X\) s.t. there exists \(A\in\mathscr{B}\)
    with \(B\subset A\) and \(\mu(A)=0\).
\end{remark}
\begin{proposition}
    The \(\sigma\)-algebra \(\bar{\mathscr{B}}\) can also be described as follows:
    \begin{align}
        \bar{\mathscr{B}} \eqdef \left\{ B\subset X: A_1\subset B\subset A_2\text{ for some }A_1,A_2\in\mathscr{B}\text{ with }\mu(A_2\backslash A_1)=0 \right\},
    \end{align}
    with \(B,A_1,A_2 \) as above, we define 
    \begin{align}
        \bar{\mu} \eqdef \mu(A_1) = \mu(A_2)
    \end{align}
    Then \(\left( X,\bar{\mathscr{B}}, \bar{\mu} \right)\) is a complete measure space.
\end{proposition}

%%% Proof %%%

\begin{definition}
    If \(\mu\) is a Borel measure on a \textbf{metric} space \((X,d)\), then the completion \(\bar{\mathscr{B}}(X)\) of the Borel \(\sigma\)-algebra
    with respect to \(\mu\) is called the \(\sigma\)-algebra of \(\mu\)-measurable sets. 
\end{definition}
\begin{remark}
    For \(\mu=\lambda_n\) on \(\mathbb{R}^n\) we talk about the \(\sigma\)-algebra of \textbf{Lebesgue measurable sets}. Instead of
    \(\bar{\lambda_n}\) we still write \(\lambda_n\) and call it the \textbf{Lebesgue measure}. A function \(f:\mathbb{R}^n\rightarrow \mathbb{C}\),
    measurable w.r.t. the \(\sigma\)-algebra of Lebesgue measurable sets is called the \textbf{Lebesgue measurable}.
\end{remark}
\emph{The following result shows that any Lebesgue measurable function coincides with a Borel function a.e.}
\begin{proposition}
    Assume \(\left( X,\mathscr{B},\mu \right)\) is a measure space and consider its completion \(\left( X,\bar{\mathscr{B}},\bar{\mu} \right)\).
    Assume \(f:X\rightarrow\mathbb{C}\) is \(\bar{\mathscr{B}}\)-measurable. Then there is a \(\mathscr{B}\)-measurable function 
    \(g:X\rightarrow\mathbb{C}\) s.t. \(f=g \ \bar{\mu}\)-a.e.
\end{proposition}