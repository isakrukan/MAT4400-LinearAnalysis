\section*{Measurable Functions \tiny{ (8, \cite{schilling2017measures})}}

A \textit{measurable function} is a measurable map $\map{u}{X}{ \R}$ from some measurable space $(X,\A)$
 to $(\R,\Borel{1})$.
They play central roles in the theory of integration. \\

We recall that $\map{u}{X}{\R} $ is $\A/\Borel{1}$-measurable if \begin{align*}
    u^{-1}(B)\in\A,\hspace{2mm}\forall B\in \Borel{1}.
\end{align*}

Moreover from a lemma from chapter 7, we actually only need to show that \begin{align*}
    u^{-1}(G)\in\A,\hspace{2mm} \forall G\in\mathcal G \text{ where } \mathcal G \text{ generates } \Borel{1}.
\end{align*} 

\begin{proposition}
    $\hspace{2mm}$
    \begin{itemize}
        \item[1] If $f,g:(X,\B)\rightarrow \mathbb C$ are measurable, then the function $f+g$, $f\cdot g$, $cf,\hspace{2mm}(c\in\mathbb C)$
         are measurable.
         \item[2] If $b:\mathbb C\rightarrow \mathbb C$ is Borel and $b:(\mathbb C,\B)\rightarrow \mathbb C$ is measurable, then $b\circ f$ is measurable.
         \item[3] If $f(x)=\lim_{n\rightarrow \infty}f_n(x),\hspace{2mm}x\in X$ and $f_n$ are measurable, then $f$ is measurable.
         \item[4] If $X=\bigcup_{n=1}^\infty A_n,\hspace{2mm} (A_n\in \B)$, $f|_{A_n}:(A_n,\B_{A_{n}})\rightarrow \mathbb C$ is measurable $\forall n$, then $f$ is measurable.
    \end{itemize}
\end{proposition}

\begin{definition}
    Given a measurable space $(X,\B)$, a measurable function $f:(X,\B)\rightarrow \C$ is called simple if 
    \begin{align*}
    f(x) = \sum_{k=1}^{N}c_k\mathds 1_{A_k}(x),    
    \end{align*}
for some $c_k\in\C$, $A_k\in \B$, where $\mathds 1$ is the characteristic function,

\begin{align*}
    \mathds 1_{A}(x)= \begin{cases} 
        1 & \text{if } x\in A \\
        0 & \text{if } x\notin A.
     \end{cases}
\end{align*}

The representation of simple function is \textbf{not} unique. We denote the standard representation of
$f$ by 
\begin{align*}
    f(x) = \sum_{n=0}^{N}z_n\mathds 1_{B_n}(x),
\end{align*}
for \(N\in\mathbb N,\hspace{1mm} z_n\in\R,\hspace{1mm} B_n\in\A\), and 
\begin{align*}
    X=\bigcup_{n=1}^N B_n, 
\end{align*}
for \(B_n\cap B_m=\emptyset,\hspace{2mm} n\neq m.\)
The set of simple functions is denoted $\mathcal E(\A)$ of $\mathcal E$. 

\end{definition}
\begin{definition}
    Assume $\mu$ is a measure on $(X,\B)$. Given a \textit{positive} simple function
    \begin{align*}
        f = \sum_{k=1}^{N}c_k\mathds 1_{A_k},\hspace{4mm} (c_k\geq 0).
    \end{align*}
    We define \begin{align*}
        \int_X fd\mu =\sum_{k=1}^{n}c_k\mu(A_k)\in [0,+\infty].
    \end{align*}
    We also denote this by $I_\mu(f)$.
\end{definition}

\begin{lemma}
    This is well defined, that is, $\int_x fd\mu$ does not depend on the presentation of the simple function $f$.
\end{lemma}

\begin{properties}
    For every positive simple function
    \begin{center}
\begin{itemize}
    \item[1] $\int_X cfd\mu = c\int_Xfd\mu,\hspace{4mm}\text{for only } c\geq 0$
    \item[2] $\int_X(f+g)d\mu = \int_Xfd\mu + \int_X gd\mu$.
\end{itemize}
\end{center}
\end{properties}

\begin{corollary}
    If $f\geq g\geq 0$ are simple functions, then 
    \begin{align*}
        \int_X fd\mu \geq \int_X g d\mu.
    \end{align*}
\end{corollary}

\begin{definition}
    If $f:X\rightarrow [0,+\infty)$ is measurable, then we define 
    \begin{align*}
        \int\limits_{X} fd\mu = \sup\left\{  \int_X gd\mu : f\geq g\geq 0,\hspace{2mm} g\text{ is simple}\right\}
    \end{align*}
\end{definition}
\begin{remark}
    This means that any measurable function can be approximated by simple functions.
\end{remark}

\begin{properties}
    Measurable functions like this have the following properties 
    \begin{itemize}
        \item[1] $\int_X cfd\mu = c\int_X fd\mu,\hspace{3mm} \forall c\geq 0. $
        \item[2] If $f\geq g\geq 0$, then $\int_X fd\mu\geq\int_X gd\mu$ for any measurable $g,f$.
        \item[3] If $f\geq 0$ is simple, then $\int_X fd\mu $ is the same value as obtained before.    
    \end{itemize}
\end{properties}

To advance in measure theory we consider measurable functions $$f:X\rightarrow [0,+\infty].$$ Measurability is understood w.r.t the $\sigma-$algebra
 $\B([0,+\infty])$ generated by $\B(\left[0,+\infty\right))$ and $\left\{+\infty\right\}$. In other words, $A\subset[0,+\infty]\in B([0,+\infty])$ iff $A\cap [0,+\infty)\in \B([0,+\infty)$.
\begin{remark}
    Hence $f:X\rightarrow [0,+\infty]$ is measurable iff $f^{-1}(A)$ is measurable $\forall A\in \B([0,+\infty))$.
\end{remark}

\begin{definition}
    For measurable functions $f_X\rightarrow [0,+\infty]$, we define \begin{align*}
        \int_X fd\mu = \sup\left\{\int_xgd\mu \hspace{1mm}:\hspace{1mm} f\geq g\geq 0\hspace{1mm}:\hspace{1mm} g \text{ is simple}\right\}\in[0,+\infty].
    \end{align*}
\end{definition}

\begin{theorem}[\color{red}\textbf{Monotone convergence theorem}\color{black}]
    Assume $(X,\B,\mu)$ is a measure space, $(f)_{n=1}^\infty$ is an increasing sequence of measurable positive functions $f_n:X\rightarrow [0,+\infty]$. Define $f(x)=\lim_{n\rightarrow \infty}f_n(x)$. 
    Then $f$ is measurable and  \begin{align*}
        \int_X fd\mu = \lim_{n\rightarrow \infty} \int_X f_nd\mu.
    \end{align*}
    
\end{theorem}

\begin{theorem}
    Assume $(X,\B)$ is a measurable space and $f:X\rightarrow [0,+\infty]$ is measurable. Then there are simple functions $g_n$, s.t. \begin{align*}
        0\leq g_1\leq g_2\leq\ldots \hspace{2mm},\hspace{2mm} g_n(x)\rightarrow f(x),\hspace{2mm}\forall x\in X.
    \end{align*}
    Moreover, if $f$ is bounded, we can choose $g_n$ s.t. the convergence is uniform, that is, \begin{align*}
        \lim_{n\rightarrow\infty}\sup_{x\in X}|g_n(x)-f(x)|=0.
    \end{align*}
\end{theorem}