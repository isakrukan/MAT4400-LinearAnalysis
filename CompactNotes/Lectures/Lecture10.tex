\section{Regularity of Measures \tiny{(App. H, \cite{schilling2017measures})}}

We let $(X,d)$ be a metric space and denote by $\mathcal O$ the open, by $\mathcal C$ the closed and $\B(X)=\sigma(\mathcal O)$ the Borel set of $X$.

\begin{definition}[\colordefinition{outer and inner regular measures}]
    A measure $\mu$ on $(X,d,\B(X))$ is called outer regular, if \begin{align}
        \mu(B) = \inf\left\{\mu(U)\omm|\omm B\subset U,\omm U \text{ open} \right\}
    \end{align}
    and inner regular, if $\mu(K)<\infty$ for all compact sets $K\subset X$ and \begin{align}
        \mu(U) = \sup\left\{\mu(K)\omm|\omm K\subset U,\omm K \text{ compact} \right\}.
    \end{align}
    A measure which is both inner and outer regular is called \textbf{regular}. We write $\mathfrak m_r^+(X)$ for the family of 
    regular measures on $(X,\B(X))$.
\end{definition}

\begin{remark}
    The space $X$ is called $\sigma$-compact if there is a sequence of compact sets $K_n\uparrow X$. A typical example of such a space 
    is a locally compact, separable metric space.
\end{remark}

\begin{theorem}
    Let $(X,d)$ be a metric space. Every finite measure $\mu$ on $(X,\B(X))$ is outer regular. If $X$ is $\sigma$-compact, then $\mu$ is also inner regular, hence regular.
\end{theorem}
\begin{theorem}
    Let $(X,d)$ be a metric space and $\mu$ be a measure on $(X,B(X))$ such that $\mu(K)<\infty$ for all compact sets $K\subset X$.
\end{theorem}
\begin{itemize}
    \item[1] If $X$ is $\sigma$-compact, then $\mu$ is inner regular.
    \item[2] If there exists a sequence $G_n\in\mathcal O,\omm G_n\uparrow X$ such that $\mu(G_n)<\infty$, then $\mu$ is outer regular.   
\end{itemize}