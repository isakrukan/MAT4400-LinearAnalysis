\section{\underline{}\(\boldsymbol{\sigma}\)-Algebras \tiny{(3, \cite{schilling2017measures})}}
\begin{definition}[\(\sigma\)-Algebra]
    A family \(\mathscr{A}\) of subsets of \(X\) with:
    \begin{enumerate}[label=(\roman*)]
        \item \(X\in\mathscr{A}\), \label{ax:sigma1}
        \item \(A\in\mathscr{A} \Rightarrow A^c \in\mathscr{A}\), \label{ax:sigma2}
        \item \(\left( A_n \right)_{n\in\mathbb{N}} \in\mathscr{A} \Rightarrow \bigcup\limits_{n\in\mathbb{N}} \)
    \end{enumerate}
\end{definition}
\begin{theorem}[and Definition]
    \quad 

    \begin{enumerate}[label=(\roman*)]
        \item The intersection of arbitrarily many \(\sigma\)-algebras in \(X\) is againg a \(\sigma\)-algebra in X.
        \item \(\text{For every system of sets }p \subset \mathscr{P}(X)\) there exists a 
        \(\text{smallest} \sigma\text{-algebra containing} p. \) This is the \(\sigma\)-algebra
        \(\text{generated by } p\), denoted \(\sigma(p)\), and 
        \(\sigma(p) \text{ is called its generator}\).
    \end{enumerate}
\end{theorem}
\begin{definition}[Borel]
    The \(\sigma\)-algebra \(\sigma(\mathcal{O})\) generated by the open sets \(\mathcal{O} = \mathcal{O}_{\mathbb{R}^n}\) of $\mathbb{R}^n$
    is called \textbf{Borel $\sigma$-algebra}, and its members are called \textbf{Borel sets} or
    \textbf{Borel measurable sets}.
\end{definition}