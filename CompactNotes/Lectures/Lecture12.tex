\section{Dense and Determining Sets \tiny{(17, \cite{schilling2017measures})}}
\begin{definition}[Dense Sets]
    A set \(\mathcal{D}\subset\mathcal{L}^p(\mu), p\in[0,\infty]\), is called \emph{dense} if for every \(u\in\mathcal{L}^p(\mu)\) there exist
    a sequence \((f_n)_{n\in\mathbb{N}}\subset\mathcal{D}\) s.t. \(\lim_{n\rightarrow\infty}\vert\vert u-f_n\vert\vert_p=0\).
\end{definition}
\begin{definition}[Support]
    The support of a function \(f\) is the set of points in \(X\) where \(f\) is non-zero:
    \begin{align*}
        \text{supp}(f) \eqdef \left\{x\in X:f(x)\neq0\right\}.
    \end{align*}
\end{definition}
\begin{theorem}
    Assume \(X,d\) is a metric space and \(\mu\) is a Borel measure that is finite on every ball \(1\leq p<\infty\). Then the space of 
    bounded continuous functions with bounded support is dense in \(\mathcal{L}^p(X,d\mu)\). Where bounded support means that \(f\) vanishes outside
    some ball.
\end{theorem}
\ifdetailed
\begin{proof}
    We want to approximate \(f\in\mathcal{L}^p(X,d\mu)\) by bounded continuous functions with bounded support. By considering separately 
    \((\text{Re}(f))_I\) and \((\text{Im}(f))_I\) we may assume that \(f\geq 0\). Then we can find simple functions \(f_n\) s.t. 
    \(0\leq f_n\leq f, f_n\rightarrow f\) pointwise. As \(\vert  f- f_n\vert^p \leq \vert f\vert^p\), by the dominated convergence theorem
    we have \(f_n\rightarrow f\in\mathcal{L}^p(X,d\mu)\). Hence, it suffices to consider simple \(f\), but then it suffices to approximate
    \(f=\pi_A\). Note that \(\pi_A\in\mathcal{L}^p(X,d\mu)\) iff \(\mu(A)<\infty\). 

    Fix \(x_0\in X\). Then \(\pi_{A\cap B_n(x_0)} \nearrow \pi_A\) pointwise, hence 
    \(\pi_{A\cap B_n(x_0)}\rightarrow \pi_A\in\mathcal{L}^p(X,d\mu)\), again by the dominated convergence theorem. 

    Therefor it suffices to consider \(A\subset B_n(x_0)\). As \(\mu\) is outer regular, we have 
    \begin{align*}
        \mu(A) = \inf_{\substack{A\subset U\subset B_n(x_0) \\ U \text{ is open}}} \mu(U).
    \end{align*}
    Note that \(\vert\vert \pi_U - \pi_A\vert\vert_p = \mu(U\backslash A)^{1/p}\). Hence, we can choose \(U_k\subset B_n(x_0)\) s.t.
    \(A\subset U_k\), \(U_k\) is open, \(\pi_{U_k}\rightarrow \pi_A\in\mathcal{L}^p(X,d\mu)\).

    Therefor it suffices to approximate \(\pi_U\) for open \(U\subset B_n(x_0)\). Consider the functions
    \begin{align*}
        f_k(x) = \frac{kd(x,U^c)}{1+kd(x,U^c)}.
    \end{align*}
    Then \(0\leq f_k\leq 1\), \(f_k\) is continuous, supported on \(\bar{U} \subset \bar{B}_n(x_0)\) and \(f_k\nearrow\pi_U\) pointwise, hence
    \(f_k\xrightarrow[k\rightarrow \infty]{ } \pi_U\in\mathcal{L}^p(X,d\mu)\).
\end{proof}
\fi 
\begin{theorem}
    Assume \((X,d)\) is a separable locally compact metric space and \(\mu\) is a Borel Measure on \(X\) s.t. \(\mu(K)<\infty \ \forall \)
    compact \(K\subset K\). Then the space \(C_c(X)\) of continuous compactly supported functions is dense in \(\mathcal{L}^p(X,d\mu)\).
\end{theorem}

Recall that the support of a function \(f\) is \(\text{supp}(f) = \{x\in X: f(x) \neq 0\}\), \emph{closed support} is the closure of 
\(\text{supp}(f)\) (i.e. boundary points
are included), often just written as \(\text{supp}(f)\), and a function is said to have \emph{compact support} if \(\text{supp}(f)\) is
\emph{compact}.

In particular, either theorem shows that if \(\mu\) is a Borel measure on \(\mathbb{R}^n\) s.t. the measure of every ball is finite, then 
\(C_c(\mathbb{R}^n)\) is dense in \(\mathcal{L}^p(\mathbb{R}^n, d\mu)\), \(1\leq p< \infty\). Later we will see that even 
\(C^{\infty}_{c}(\mathbb{R}^n)\) is dense in \(L^p(\mathbb{R}^n, d\mu)\).
\begin{remark}
    These results do not extend to \(p=\infty\) in general. 

    For \(\mu=\lambda_n\) we write simply \(\mathcal{L}^p(\mathbb{R}^n)\).
\end{remark}
\begin{remark}
    Theorem 17.8 in the book is WRONG. For example, \(X=\mathbb{Q}\) with the usual metric is \(\sigma\)-compact, supports nonzero finite measure,
    but \(C_c(\mathbb{Q})=0\).
\end{remark}

\section*{Modes of Convergence \protect\\ \tiny{(mixture of ex. 11.12 and ch. 22 p. 258-261. in \cite{schilling2017measures})}}
\ifdetailed
\begin{definition}[convergence in meassure]
    A sequence of measurable functions \(u_n:X\rightarrow \bar{\mathbb{R}}\) \emph{converges in measure} if
    \begin{align*}
        \forall\epsilon>0\forall A\in\mathscr{A},\mu(A)<\infty:\lim\limits_{n\rightarrow\infty} 
        \mu\left(\left\{\vert u_n - u\vert > \epsilon\right\}\cap A\right) = 0
    \end{align*}
    holds for some \(u\in\mathcal{M}(\mathscr{A})\). We write \(\mu\)-\(\lim_{n\rightarrow\infty u_n = u}\) or 
    \(u_n\xrightarrow[ ]{\mu} u\).
\end{definition}
\fi
Assume \(\left(X,\mathscr{B},\mu\right)\) is a measure space. Given measurable functions \(f_n,f:X\rightarrow\mathbb{C}\), recall that
\begin{align*}
    f_n\rightarrow f \text{ a.e.}
\end{align*}
means that \(f_n(x)\xrightarrow[n\rightarrow\infty]{ } f(x)\) for all x outside a set of measure zero. 
\begin{theorem}[\ifcolor \textbf{\color{red}Egorov}\color{black}\else Egorov\fi]
    Assume \(\mu(X)<\infty\) and \(f_n\rightarrow f\) a.e. Then, \(\forall\epsilon>0\), there exists \(X_{\epsilon} \in\mathscr{B}\)
    s.t. \(\mu(X_{\epsilon})<\epsilon\) and \(f_n\rightarrow f\) uniformly on \(X\backslash X_{\epsilon}\).
\end{theorem}
In addition to pointwise and uniform convergence we also consider the following: 

\(f_n\rightarrow f\) in the \emph{p-th mean} if 
\(\vert \vert f_n-f\vert\vert_p\xrightarrow[n\rightarrow\infty]{ } 0\). For \(p=1\) we say \emph{in mean}, for \(p=2\) we say in 
quadratic mean. 

\(f_n\rightarrow f\) \emph{in measure } if \(\forall\epsilon>0\) we have
\begin{align*}
    \mu\left(\left\{x:\vert f_n(x)-f(x)\vert \geq \epsilon\right\}\right) \xrightarrow[n\rightarrow \infty]{ } 0.
\end{align*}
\begin{theorem}[Lemma 22.4 in the book?]
    Assume \(\left(X, \mathscr{B}, d\mu\right)\) is a measure space, \(1\leq p <\infty\), \(f_n,f:X\rightarrow \mathbb{C}\) are measurable 
    functions. Then
    \begin{enumerate}[label=(\roman*)]
        \item If \(f_n\rightarrow f\) in the p-th mean, then \(f_n\rightarrow f\) in measure.
        \item If \(f_n\rightarrow f\) in measure, then there is a subsequence \((f_{n_k})^{\infty}_{k=1}\) s.t. \(f_{n_k}\rightarrow f\) a.e.
        \item If \(f_n\rightarrow f\) a.e. and \(\mu(X)<\infty\), then \(f_n\rightarrow f\) in measure. 
    \end{enumerate}
    In particular, if \(f_n\rightarrow f\) in the p-th mean, then \(f_{n_k}\rightarrow f\) a.e. for a subsequence \((f_{n_k})_k\).
\end{theorem}