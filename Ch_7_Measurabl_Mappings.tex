\documentclass{article}


% allows special characters (including æøå)
\usepackage[utf8]{inputenc}
%\usepackage[english]{babel}

\usepackage{subfiles}
\usepackage{physics,amssymb}  % mathematical symbols (physics imports amsmath)
\include{amsmath}
\usepackage{graphicx}         % include graphics such as plots
\usepackage{xcolor}           % set colors
\usepackage{hyperref}         % automagic cross-referencing (this is GODLIKE)
\usepackage{listings}         % display code
\usepackage{subfigure}        % imports a lot of cool and useful figure commands
\usepackage{float}
%\usepackage[section]{placeins}
\usepackage{algorithm}
\usepackage[noend]{algpseudocode}
\usepackage{subfigure}
\usepackage{tikz}
\usepackage{cleveref} % for \cref
\usepackage{enumitem} % to enumerate with a), b), ... : [label=(\alph*)] 
\usepackage{cancel}
\usepackage{slashed}
\usepackage{amsthm}

\newtheorem{theorem}{Theorem}[section]
\newtheorem{lemma}[theorem]{Lemma}
\newtheorem{corollary}{Corollary}[theorem]
\newtheorem{prop}{Proposition}
\newcommand{\eqdef}{\mathrel{\mathop:}=}
\theoremstyle{definition}
\newtheorem{definition}{Definition}[theorem]
\usepackage{stmaryrd}

\usetikzlibrary{quantikz}

% defines the color of hyperref objects
% Blending two colors:  blue!80!black  =  80% blue and 20% black

\lstset{frame=tb,
  language=c++,
  aboveskip=3mm,
  belowskip=3mm,
  showstringspaces=false,
  columns=flexible,
  basicstyle={\small\ttfamily},
  numbers=none,
  numberstyle=\tiny\color{gray},
  keywordstyle=\color{blue},
  commentstyle=\color{dkgreen},
  stringstyle=\color{orange},
  breaklines=true,
  breakatwhitespace=true,
  tabsize=4
}
\hypersetup{
    colorlinks,
    linkcolor={red!50!black},
    citecolor={blue!50!black},
    urlcolor={blue!80!black}}
%% USEFUL LINKS:
%%
%%   UiO LaTeX guides:        https://www.mn.uio.no/ifi/tjenester/it/hjelp/latex/
%%   mathematics:             https://en.wikibooks.org/wiki/LaTeX/Mathematics

%%   PHYSICS !                https://mirror.hmc.edu/ctan/macros/latex/contrib/physics/physics.pdf

%%   the basics of Tikz:       https://en.wikibooks.org/wiki/LaTeX/PGF/Tikz
%%   all the colors!:          https://en.wikibooks.org/wiki/LaTeX/Colors
%%   how to draw tables:       https://en.wikibooks.org/wiki/LaTeX/Tables
%%   code listing styles:      https://en.wikibooks.org/wiki/LaTeX/Source_Code_Listings
%%   \includegraphics          https://en.wikibooks.org/wiki/LaTeX/Importing_Graphics
%%   learn more about figures  https://en.wikibooks.org/wiki/LaTeX/Floats,_Figures_and_Captions
%%   automagic bibliography:   https://en.wikibooks.org/wiki/LaTeX/Bibliography_Management  (this one is kinda difficult the first time)
%%   REVTeX Guide:             http://www.physics.csbsju.edu/370/papers/Journal_Style_Manuals/auguide4-1.pdf
%%
%%   (this document is of class "revtex4-1", the REVTeX Guide explains how the class works)


%%%%%%%%%%%%% CREATING THE pdf FILE USING LINUX IN THE TERMINAL %%%%%%%%%%%%%
%% pdflatex filename.tex && filename.tex && open filename.pdf

%%%%%%%%%%%%%%%%%%%%%%%%%% USING FOOTNOTE COMMAND: %%%%%%%%%%%%%%%%%%%%%%%%%%
    %% pdflatex filename.tex && bibtex filename.tex && pdflatex filename.tex && pdflatex filename.tex && open filename.pdf


\setcounter{section}{6}
\setcounter{theorem}{6}

\begin{document}
	
	\title{Measurable Mappings}
	\maketitle
	\date{}



    We consider maps $T:X\rightarrow X'$ between two measurable spaces $(X,\mathcal A)$ and $(X',\mathcal A')$ which respects the measurable 
    structurs, the $\sigma$-algbras on $X$ and $X'$. These maps are useful as we can transport a measure $\mu$, defined on $(X,\mathcal A)$, to $(X',\mathcal A')$.

    \section{Measurable Mappings}

    \begin{definition}
        Let $(X,\mathcal A)$, $(X',\mathcal A')$ b measurable spaces. A map $T:X\rightarrow X'$ is called $\mathcal A/\mathcal A'$-measurable if the pre-imag of every measurable set is a measurable set: 
        \begin{align}
        T^{-1}(A')\in \mathcal A,\hspace{3mm} \forall A'\in\mathcal A'.    
        \end{align}

    \end{definition}
    \begin{itemize}
        \item A $\mathcal B(\mathbb R^n)/\mathcal B(\mathbb R^m)$ measurable map is often called a Borel map. 
        \item The notation $T:(X,\mathcal A)\rightarrow (X',\mathcal A')$ is often used to indicate measurability of the map $T$.
    \end{itemize}

    \begin{lemma}
        Let $(X,\mathcal A)$, $(x',\mathcal A')$ be measurable spaces and let $\mathcal A' = \sigma(\mathcal G')$. Then $T:X\rightarrow X'$ is $\mathcal A/\mathcal A'$-measurable iff $T^{-1}(\mathcal G')\subset \mathcal A$, i.e. if 
        \begin{align}
            T^{-1}(G')\in\mathcal A,\hspace{2mm} \forall G'\in\mathcal G'.
        \end{align} 
    \end{lemma}


	\begin{theorem}

        Let $(X_i,\mathcal A_i),\hspace{2mm}i=1,2,3$, be measurable spaces and $T:X_1\rightarrow X_2$, $S:X_2\rightarrow X_3$ be $\mathcal A_1/\mathcal A_2$ and $\mathcal A_2/\mathcal A_3$-measurable maps respectivly. Then $S\circ T:X_1\rightarrow X_3$ is $\mathcal A_1/\mathcal A_3$-measurable. 

	\end{theorem}

    \begin{definition}
        \textbf{(and lemma)}
        Let $(T_i)_{i\in I}$, $T_I:X\rightarrow X_i$, be arbitrarily many mappings from the same space X into measurable spaces $(X_i,\mathcal A_i).$ The smallest $\sigma$-algebra on $X$ that makes all $T_i$ simultanously measurable is 
        \begin{align}
            \sigma(T_i:i\in I) := \sigma \left( \bigcup_{i\in I}T_i^{-1}(\mathcal A_i)\right)
        \end{align}

    \end{definition}
	
    \begin{theorem}
        Let $(X,\mathcal A)$, $(X',\mathcal A')$ be measurable spaces and $T:X\rightarrow X'$ be an $\mathcal A/\mathcal A'$-measurable map. For every measurable $\mu$ on $(X,\mathcal A)$, 
        \begin{align}
            \mu'(A') := \mu (T^{-1}(A')),\hspace{3mm} A'\in \mathcal A',
        \end{align}
        defines a measure on $(X',\mathcal A')$.
    \end{theorem}

    \begin{definition}
        The measure $\mu'(\cdot)$ in the above theorem is called the push forward or image measure of $\mu$ under $T$ and it is denoted as 
        $T(\mu)(\cdot)$, $T_{*\mu}(\cdot)$ or $\mu\circ T^{-1}(\cdot)$.
    \end{definition}

    \begin{theorem}
        If $T\in \mathbb R^{n\times n} $ is an orthogonal matrix, then $\lambda^n = T(\lambda^n).$
    \end{theorem}

    \begin{theorem}
        Let $S\in\mathbb R^{n\times n}$ be an invertible matrix. Then 
        \begin{align}
        S(\lambda^n) = |\det s^{-1}|\lambda^n = |\det S|^{-1}\lambda^n.    
        \end{align}
    \end{theorem}

    \begin{corollary}
        Lebesgue measure is invariant under motions: $\lambda^n=M(\lambda^n)$ for all motions $M$ in $\mathbb R^n$. In particular, congruent sets have the same measure. Two sets of points are called congruent if, and only if, one can be transformed into the other by an isometry
    \end{corollary}
\end{document}
